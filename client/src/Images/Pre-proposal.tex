% Options for packages loaded elsewhere
\PassOptionsToPackage{unicode}{hyperref}
\PassOptionsToPackage{hyphens}{url}
%
\documentclass[
]{article}
\usepackage{lmodern}
\usepackage{amssymb,amsmath}
\usepackage{ifxetex,ifluatex}
\ifnum 0\ifxetex 1\fi\ifluatex 1\fi=0 % if pdftex
  \usepackage[T1]{fontenc}
  \usepackage[utf8]{inputenc}
  \usepackage{textcomp} % provide euro and other symbols
\else % if luatex or xetex
  \usepackage{unicode-math}
  \defaultfontfeatures{Scale=MatchLowercase}
  \defaultfontfeatures[\rmfamily]{Ligatures=TeX,Scale=1}
\fi
% Use upquote if available, for straight quotes in verbatim environments
\IfFileExists{upquote.sty}{\usepackage{upquote}}{}
\IfFileExists{microtype.sty}{% use microtype if available
  \usepackage[]{microtype}
  \UseMicrotypeSet[protrusion]{basicmath} % disable protrusion for tt fonts
}{}
\makeatletter
\@ifundefined{KOMAClassName}{% if non-KOMA class
  \IfFileExists{parskip.sty}{%
    \usepackage{parskip}
  }{% else
    \setlength{\parindent}{0pt}
    \setlength{\parskip}{6pt plus 2pt minus 1pt}}
}{% if KOMA class
  \KOMAoptions{parskip=half}}
\makeatother
\usepackage{xcolor}
\IfFileExists{xurl.sty}{\usepackage{xurl}}{} % add URL line breaks if available
\IfFileExists{bookmark.sty}{\usepackage{bookmark}}{\usepackage{hyperref}}
\hypersetup{
  hidelinks,
  pdfcreator={LaTeX via pandoc}}
\urlstyle{same} % disable monospaced font for URLs
\usepackage{longtable,booktabs}
% Correct order of tables after \paragraph or \subparagraph
\usepackage{etoolbox}
\makeatletter
\patchcmd\longtable{\par}{\if@noskipsec\mbox{}\fi\par}{}{}
\makeatother
% Allow footnotes in longtable head/foot
\IfFileExists{footnotehyper.sty}{\usepackage{footnotehyper}}{\usepackage{footnote}}
\makesavenoteenv{longtable}
\setlength{\emergencystretch}{3em} % prevent overfull lines
\providecommand{\tightlist}{%
  \setlength{\itemsep}{0pt}\setlength{\parskip}{0pt}}
\setcounter{secnumdepth}{-\maxdimen} % remove section numbering

\author{}
\date{}

\begin{document}

\hypertarget{introduction}{%
\subsubsection{Introduction}\label{introduction}}

Although surveillance cameras are allowed to identify the fast events,
it is not sufficient to full fill human requirement. In that case real
time event detection is more critical in present day world. Especially
instant detection of anomalies in usual surrounding is very important
for security concerns and other time depended situations. Manual
monitoring of fast event in surveillance camera may be useless because
of high time consumption, lack of detecting ability of humans and delay
to take necessary action. Therefore, there is a high demand for real
time event detecting system, but that is not famous in commercially. In
this situation, computerize system of artificial intelligence with
surveillance camera for detecting and indicating anomalies may be a good
solution and it will improve the effectiveness and efficiency of
surveillance camera system and other possible cases. Among the thousands
of possible use cases and applications this system well fit and will
produce high efficiency output for home surrounding because of low
complexity.

\hypertarget{background}{%
\subsubsection{Background}\label{background}}

\textbf{Problem statement}

Surveillance cameras are usually used in everywhere to identify the
incidents. But problem is the, owner of the surveillance camera has to
identify the particular unusual event by manually and has to checkout it
by walking through whole video. When manually detecting the unusual
event, it may be too delay to take action or sometimes video data may be
removed from the storage. The existing solution for detect and indicate
anomalies in real time such as motions sensor, etc. are failed because
of they are not intelligence to identify normal and abnormal events. If
not lot of employers have been assigned to monitor real time incidents.
This may be expensive and less effective due to human inabilities

In that case anomalies detection of surveillance camera in real time is
coming into the picture. Producing real time anomalies detection system
for CCTV camera system, above problems will be solve very effectively.

\textbf{Objective and Scope}

The main goal of this project is the implementing a system for detecting
real time anomalies in surveillance camera. This goal is achieved by
completing following objectives.

\textbf{Objectives}

\begin{enumerate}
\def\labelenumi{\arabic{enumi}.}
\item
  Input data to real time processing
\end{enumerate}

\begin{itemize}
\item
  Surveillance video footage from cameras
\end{itemize}

\begin{enumerate}
\def\labelenumi{\arabic{enumi}.}
\setcounter{enumi}{1}
\item
  Preprocessing video frames
\end{enumerate}

\begin{itemize}
\item
  Convert frames to grayscale
\item
  Remove noise or blur from frames
\end{itemize}

\begin{enumerate}
\def\labelenumi{\arabic{enumi}.}
\setcounter{enumi}{2}
\item
  Identifying suitable feature extraction method
\end{enumerate}

\begin{itemize}
\item
  Extract edges, shapes and patterns from frames
\end{itemize}

\begin{enumerate}
\def\labelenumi{\arabic{enumi}.}
\setcounter{enumi}{3}
\item
  ML model training for classify the normal and anomalies
\end{enumerate}

\begin{itemize}
\item
  Train a machine learning model of neural network
\end{itemize}

\begin{enumerate}
\def\labelenumi{\arabic{enumi}.}
\setcounter{enumi}{4}
\item
  Model evaluation
\end{enumerate}

\begin{itemize}
\item
  Evaluate the model's performance on a separate test set
\end{itemize}

\begin{enumerate}
\def\labelenumi{\arabic{enumi}.}
\setcounter{enumi}{5}
\item
  Real-time detection
\end{enumerate}

\begin{itemize}
\item
  Use the trained model to classify new frames from the surveillance
  video as normal or abnormal in real-time
\end{itemize}

\begin{enumerate}
\def\labelenumi{\arabic{enumi}.}
\setcounter{enumi}{6}
\item
  Implement an alert generation system to generate an alert if an
  abnormal event is detected
\end{enumerate}

\textbf{Scope}

\begin{enumerate}
\def\labelenumi{\arabic{enumi}.}
\item
  In the first stage, single camera data will feed into the system for
  processing.
\item
  Quality of the video in CCTV camera may be reduced to grayscale color
  scheme for reliable fast operation
\item
  Frame of real time video will be fixed
\item
  All the experiments will be done focusing home environment in the
  Faculty of Engineering university of Ruhuna
\end{enumerate}

\textbf{Methodology}

\textbf{Work plan and budget}

Work Plan

The estimated work plan and the milestones for the project can be
mentioned as follows. Letters A,B and C mentioned in below tables
represent,

A -- Basnayaka B.C.H

B -- Kumaradasa

C- Madusanka I.K.

MILESTONE 01: PROJECT TOPIC SELECTION

\begin{longtable}[]{@{}ll@{}}
\toprule
Tasks &\tabularnewline
\midrule
\endhead
Task 1.1 & Select the topics which is suitable for us\tabularnewline
Task 1.2 & Read the given project proposal description\tabularnewline
Task 1.3 & Discussed with the supervisors about project
objectives\tabularnewline
Task 1.4 & Select one topic and get approval\tabularnewline
\bottomrule
\end{longtable}

Table : Project topic selection

MILESTONE 02: STUDY ABOUT THE PROJECT BACKGROUND

\begin{longtable}[]{@{}ll@{}}
\toprule
Tasks &\tabularnewline
\midrule
\endhead
Task 2.1 & Understand the background\tabularnewline
Task 2.2 & Literature survey on real time event detection in
surveillance cameras\tabularnewline
Task 2.3 & Understand the existing systems and possible
technologies\tabularnewline
Task 2.4 & Understand the next steps of existing
solutions\tabularnewline
\bottomrule
\end{longtable}

Table : Studying about the background of the project

MILESTONE 03 : PROJECT PROPOSAL SUBMISSION

\begin{longtable}[]{@{}ll@{}}
\toprule
Tasks &\tabularnewline
\midrule
\endhead
Task 3.1 & Finalize the structure of the project proposal\tabularnewline
Task 3.2 & Complete the project proposal\tabularnewline
Deliverables &\tabularnewline
D 3.1 & Understand the next steps of existing solutions\tabularnewline
\bottomrule
\end{longtable}

Table : Project proposal submission

\end{document}
